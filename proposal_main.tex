\documentclass[10pt]{article}
\usepackage[utf8]{inputenc}
\usepackage{graphicx}
\usepackage{amsmath}
\usepackage{amssymb}
\usepackage{geometry}
\usepackage{setspace}
\usepackage{abstract}
\usepackage{titlesec}
\usepackage[backend=biber,style=apa]{biblatex}% Changed style to apa
\usepackage[hidelinks]{hyperref}
\PassOptionsToPackage{hyphens}{url} % Pass hyphens option to url package via hyperref
\usepackage{fancyhdr} % For custom headers and footers
\usepackage{lastpage} % For page numbering in the footer
\usepackage{xcolor} % For text color
\usepackage{tabularx} % Tabularx package for better table formatting
\usepackage{longtable}% Long table package for tables that span multiple pages

% Set marginparwidth for todonotes
\setlength{\marginparwidth}{2cm} % Adjust the marginparwidth
\usepackage{todonotes} % TODOs
\usepackage{tcolorbox} % For colored boxes
\usepackage{caption} % For custom captions

% Fonts and typography
\usepackage{helvet} % Load the Helvetica font package
\renewcommand{\familydefault}{\sfdefault} % Set sans-serif as the default font family

% Page setup
\geometry{margin=1in}
\setstretch{1.2}
\titleformat{\section}{\bfseries\Large}{\thesection}{1em}{}
\titleformat{\subsection}{\bfseries\large}{\thesubsection}{1em}{}
\titleformat{\subsubsection}{\bfseries\normalsize}{\thesubsubsection}{1em}{}

% Add bibliography file
\addbibresource{references.bib}

% Customization to remove url date and format URLs and DOIs
\AtEveryBibitem{%
  \ifentrytype{online}{%
    \clearfield{urldate}%
    \clearfield{note}%
  }{}%
  \ifentrytype{article}{%
    \clearfield{urldate}%
  }{}%
}

\renewbibmacro*{doi+eprint+url}{%
  \printfield{doi}%
  \newunit\newblock%
  \iffieldundef{doi}{%
    \usebibmacro{eprint}%
    \newunit\newblock%
    \usebibmacro{url+urldate}}%
    {}%
}

% Customize the caption format
\captionsetup[figure]{
    labelfont=bf,           % Bold font for the label
    labelsep=space           % Use a space as the separator
}
\captionsetup[table]{
    labelfont=bf,           % Bold font for the label
    labelsep=space           % Use a space as the separator
}

% Change font size of bibliography entries
\renewcommand{\bibfont}{\fontsize{7pt}{9pt}\selectfont}% Set font size to 7pt, maybe too small. Change to 8pt if needed. Also adjust actual text size?

% Title
\title{Visual Cortical Prostheses: Bridging Technology, AI and Human Vision for the Future}
\author{
  Marc J. Posthuma\\
  Student Number: 4413105\\
  \texttt{marc.posthuma@ru.nl}\\
  \\
  Radboud University\\
  Supervisor: dr.\ F.\ Zeldenrust\\
  Department of Neurophysics, Donders Centre for Neuroscience
}
\date{\today}

% Adjust page geometry to balance header and footer
\geometry{
  a4paper,
  left=20mm,
  right=20mm,
  top=30mm,
  bottom=30mm,
  headheight=60.50554pt, % Set the head height
  headsep=10pt, % Space between header and text
  footskip=30pt % Space between text and footer
}

% Custom footrule commands
\newcommand{\blackfootrule}{%
  \color{black}\makebox[\headwidth]{\rule[0.5ex]{\headwidth}{0.3pt}}%
}

\newcommand{\grayfootrule}{%
  \color{gray}\makebox[\headwidth]{\rule[0.5ex]{\headwidth}{0.3pt}}%
}

% Define fancyhdr styles
\fancypagestyle{firstpage}{
  \fancyhf{}
  \fancyhead[L]{\includegraphics[width=5cm, keepaspectratio]{imgs/RU_logo_NL_cropped.png}}
  \fancyhead[R]{\fontsize{10}{12}\selectfont \textbf{Research Proposal} \\ NWI-BM-RESPROP}
  \fancyfoot[L]{\fontsize{8}{10}\selectfont \textcolor{gray}{Radboud University}}
  \fancyfoot[C]{\fontsize{8}{10}\selectfont \textcolor{gray}{\thepage\ of~\pageref{LastPage}}}
  \fancyfoot[R]{\fontsize{8}{10}\selectfont \textcolor{gray}{July 2024}}
  \renewcommand{\footrulewidth}{0.3pt}
  \renewcommand{\footrule}{\blackfootrule}
}

\fancypagestyle{rest}{
  \fancyhf{}
  \fancyhead[L]{\fontsize{8}{10}\selectfont \textcolor{gray}{Research Proposal}}
  \fancyhead[R]{\fontsize{8}{10}\selectfont \textcolor{gray}{Visual Cortical Prostheses: Bridging Technology, AI and Human Vision for the Future}}
  \fancyfoot[L]{\fontsize{8}{10}\selectfont \textcolor{gray}{Radboud University}}
  \fancyfoot[C]{\fontsize{8}{10}\selectfont \textcolor{gray}{\thepage\ of~\pageref{LastPage}}}
  \fancyfoot[R]{\fontsize{8}{10}\selectfont \textcolor{gray}{July 2024}}
  \renewcommand{\footrulewidth}{0.3pt}
  \renewcommand{\footrule}{\grayfootrule}
}

% Redefine the abstract environment to use "Summary" instead of "Abstract"
\makeatletter
\renewenvironment{abstract}{%
    \if@twocolumn%
      \section*{\abstractname}%
    \else
      \begin{center}%
        {\bfseries \large\abstractname\vspace{-.5em}\vspace{\z@}}%
      \end{center}%
      \quotation\small % Ensures the text is smaller
    \fi}
    {\if@twocolumn\else\endquotation\fi}%
\renewcommand{\abstractname}{Summary}
\makeatother

% Document
\begin{document}

\pagestyle{plain}% Default plain page style for the list of TODOs
\listoftodos% Remove this line before submission
\clearpage%To make sure the todo list is on a separate page without fancyhdr

\newpage% To make sure the todo list is on a separate page without fancyhdr

% Title and abstract
\maketitle
\thispagestyle{firstpage} % Apply first page style after title
\begin{abstract}
  \noindent Visual cortical prostheses represent a revolutionary technology
  within the field of neuro\-prosthetics, aimed at restoring vision for
  individuals with visual impairments through direct neural interfaces. This
  proposal\ldots ABSTRACT WORK IN PROGRESS!!!
\end{abstract}
\textbf{Keywords:} Visual cortical prostheses, neuroprosthetics, artificial intelligence, phosphene patterns, real-time image processing
\vspace{1cm}

% Ensure first page style is applied until the first page is full
\thispagestyle{firstpage}
\section*{Introduction}\label{sec:intro}
Globally, blindness affects millions of people, with estimates rising from over
30 million in 2013 to 43.3 million in
2020~\parencite{stevensGlobalPrevalenceVision2013,bourneTrendsPrevalenceBlindness2021}.
For certain types of blindness, visual prosthetics present a promising avenue
for restoring rudimentary vision through electrical stimulation of the visual
system. In the past decade, significant attempts have already been made in early
systems that focus on retinal and optic nerve implants, such as the exemplary
FDA-approved ARGUSII retinal system by Second Sight
Medical~\parencite{hoLongTermResultsEpiretinal2015}. However, these systems do
not provide a solution for individuals who have damage to structures upstream in
the visual pathway.

Visual cortical prostheses provide a novel approach to stimulating the brain by
interface directly with the brain's visual
cortex~\parencite{deruytervansteveninckRealworldIndoorMobility2022}.
These devices convert visual information from the environment into neural
signals that can be processed by the brain.

The core technology involves the generation of phosphenes—perceived spots of
light resulting from electrical stimulation of the visual
cortex~\parencite{vandergrintenBiologicallyPlausiblePhosphene2024}. However,
organizing these phosphenes into coherent and interpretable visual patterns
remains a significant challenge.

In order to optimize these phosphene patterns for more accurate representations
of a user's surroundings, artificial intelligence (AI) can be leveraged in the
form of deep learning algorithms. Algorithms such as Convolutional Neural
Networks (CNNs) have already been proven to be effective for image processing of
static objects in recent work
by~\textcite{deruytervansteveninckEndtoendOptimizationProsthetic2022}.

The phosphene simulator described in the article is implemented in Python,
utilizing the PyTorch deep learning library for its computational capabilities.
It operates by translating electrical stimulation parameters into an estimated
phosphene perception, taking into account the history of stimulation to ensure
accuracy. The simulator initializes with electrode locations on a flattened
cortical map of the primary visual cortex (V1), using the reverse wedge-dipole
visuotopic model to map these locations to the user's visual field. This model
accounts for the eccentricity and azimuth in the visual field, controlling
various parameters to ensure realistic proportions in cortical distance.

To determine the size of the phosphenes, the simulator uses models that estimate
current spread from the electrodes, incorporating factors like stimulation
current and cortical magnification. The appearance and brightness of phosphenes
are modeled with a sigmoidal activation function, taking into account the
activation threshold, which introduces variability between electrodes. Temporal
dynamics are managed through a memory trace that adjusts phosphene brightness
based on stimulation history, incorporating decay and input effects to simulate
accommodation. Each frame, the simulator processes stimulation parameters,
estimates phosphene characteristics, and renders these effects on a visual field
map, summing them to produce the final simulated prosthetic percept. This
biologically grounded model aims to enhance the realism and efficacy of
simulated prosthetic vision, facilitating the optimization of cortical visual
prosthetic systems.

Deep learning can enhance these simulators by integrating them into end-to-end
optimization pipelines. In such systems, a CNN can be used to process input
images or video frames and generate appropriate electrical stimulation
parameters. The simulator then creates a visual representation based on these
parameters, which another CNN evaluates by attempting to reconstruct the
original input image. The entire system is trained through backpropagation,
where the error in the reconstructed image is used to update the parameters of
the networks, optimizing the stimulation parameters for better visual
perception.

This deep learning approach allows for the dynamic optimization of stimulation
patterns, potentially improving the quality and functionality of prosthetic
vision. It enables the simulator to adapt to complex visual stimuli and optimize
the information encoded in the phosphenes, making the simulated vision more
useful for real-life applications.


\pagestyle{rest} % Apply rest page style from here

\section*{Research}\label{sec:research}
\subsection*{Objective}\label{subsec:objective}
This study aims to develop a novel approach to generating phosphene patterns in
a way that dynamics environments can be visualized in real-time. These dynamic
systems as of yet, do not exist and are crucial for the development of visual
aid systems that can adapt to more complex and real-world scenarios. If
successful, these improved prosthetic systems hold potential of increasing the
blind user's experience dramatically.

\subsection*{Approach}\label{subsec:approach}
Explain the experiments and why they are important. Concretely explain how they
will be performed.

\todo[inline]{Explain the three phases of the study.}

\section*{Innovation}\label{sec:innovation}
Explain how the research is innovative and how it will contribute to the field.
Briefly go over the cutting-edge technology involved in the study. Maybe go
over the caveats and limitations of previous work and how this study will find
solutions.

\section*{Future Impact}\label{sec:impact}
Explain the main goal of the study and why future research will be beneficial.

\section*{Timetable}\label{sec:timetable}
The Gannt-chart goes here with a brief explanation of the different phases.

\printbibliography%

\section*{Rebuttal}\label{sec:rebuttal}
After the first review of the draft, the proposal was revised to address the
reviewers' comments. The main changes will be listed below:

\end{document}